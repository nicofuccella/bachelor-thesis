\chapter*{Sommario}
\pagenumbering{gobble}
ECLAIR è un potente strumento per la verifica del software focalizzato sullo sviluppo di sistemi high-integrity, safety-critical e security-critical. ECLAIR è pensato per funzionare sia sul desktop che sul server per analizzare interi progetti al fine di rilevarne eventuali difetti. Seguendo la tendenza verso il cosiddetto ``shift-left'' (che consiste nell'anticipare le attività di verifica già dalle prime fasi dello sviluppo), il progetto descritto in questa tesi ha l'obiettivo di fornire una prova di fattibilità di un ``sistema di verifica con feedback immediato'' usando ECLAIR. Questo al fine di dare feedback agli sviluppatori in merito alla correttezza del loro codice durante la stesura del programma, direttamente dall'IDE che già utilizzano per lo sviluppo e senza subire rallentamenti.
Una delle sfide da affrontare in un progetto di questo tipo è garantire la compatibilità con i vari IDE ed editor. La proliferazione di questi ambienti di sviluppo, ciascuno con i suoi standard e approcci consigliati per la realizzazione di estensioni, rende necessaria una separazione tra l'interfaccia dell'estensione/plugin dell'IDE e le funzionalità prettamente legate all'analisi. Il Language Server Protocol si pone come obiettivo precisamente questo: definendo un protocollo standardizzato per le comunicazioni tra un Language Server e l'ambiente di sviluppo, un solo Language Server può essere utilizzato per vari ambienti di sviluppo senza apportare modifiche.
Il passaggio dal tradizionale approccio alla fruizione dei risultati dell'analisi statica a questa modalità ``con feedback immediato'' presenta altre sfide: il tempo di analisi diventa un fattore determinante e la realizzazione di una soluzione richiede l'impiego di alcune tecniche di analisi piuttosto sofisticate (per esempio incrementalità e parallelismo).
Abbiamo realizzato un'implementazione che costituisce una prova di fattibilità dell'integrazione di ECLAIR in alcuni ambienti di sviluppo usando il Language Server Protocol. Questa tesi ha l'obiettivo di descrivere le idee sottostanti, le sfide affrontate, le possibili soluzioni esplorate durante l'implementazione del prototipo, gli insegnamenti appresi, e i futuri sviluppi che, a nostro avviso, hanno il potenziale di trasformare il prototipo in uno strumento di sviluppo software a pieno titolo.
